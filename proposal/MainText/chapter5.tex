
\chapter{Preliminary results} \label{ch-5}

It is part of the student's training in research to prepare a concise, rigorous, and scholarly thesis proposal and present it in the correct format. There is no strict length requirement for the thesis proposal. It is anticipated that most students will need 8,000-10,000 words (about twenty pages of text) to adequately explain the motivation and goals of their project, review the relevant literature, and describe progress to date. However, concise proposals are encouraged, and a proposal of 5,000 words, which covered all these points, would be perfectly acceptable. 

The complete doctoral thesis proposal document must be submitted to the Graduate School by the due date as nominated by the Dean (an example of the standard deadlines relating to examinations activities is included above). Earlier submission may be required in order to provide the thesis proposal to the examination panel no later than four weeks (28 days) prior to the oral defense. An emergency exception to the standard due date deadline can be granted by the Dean on the basis of a written request from the supervisor.

The following descriptions are sections that must be included in the proposal.

\textbf{Front page}: use the one provided in this template, after changing the values like names in the file \texttt{Preamble/mydefinitions.tex}.

\textbf{Abstract}: There should be a single paragraph of not more than 500 words, which concisely summarizes the entire proposal, written in the file \texttt{Preamble/ mydefinitions.tex}.

\textbf{Bibliography}: The bibliography should include all references cited in the text and should not include references that have not been cited. In preparing the bibliography, students may use any of the conventional styles of referencing that include the titles of articles, such as the Harvard, Vancouver or ACS systems. However, the style chosen must be used consistently and correctly throughout, both for in-text citations, and formatting of bibliographic entries. We recommend using BibTeX or BibLaTeX and through the file \texttt{Preamble/Thesis\_bibliography.bib} and referencing citations like this \cite{Lee98, Muc10, Kra27}. 

\textbf{Appendices}: These are optional and should only be used if necessary.

The main text of the proposal should contain the following sections.

\section{Introduction and Literature Review}

This should include a statement of the problem, the overall aims, and background to the research including a review of relevant existing work (literature review). The literature review should be a concise, scholarly review of the literature explaining the background to the proposed research. The review should provide the context for the aims of the proposed research in relation to existing work on the topic.

\section{Research Plan}

 This should begin with the specific aims of the research and provide a concrete plan for completion of the research including the design and methods. This section should include an explanation of how the methods will address the aims and the significance of the results for the field.

\section{Progress Report}

This should be a report on the research achievements of the student in the laboratory of the proposed supervisor during Preliminary Thesis Research. The report should not duplicate material previously submitted for evaluation as part of a previous degree, but may include work completed during rotations at OIST. The report may include examples of results obtained with the methods proposed. It is understood that results may not be available in projects requiring, for example, development of methods, sample preparation, or recruitment of participants, in which case other evidence of progress should be reported.


