\chapter{Proposed method} \label{ch-method}

\section{Variance deconvolution for uncertainty quantification}
This section outlines the methodology for developing robust theory for an accurate, efficient, and broadly applicable variance deconvolution estimator for uncertainty quantification with stochastic solvers. It also outlines plans for demonstrating the estimator's practical use. 

\subsection{Building the estimator}
We start building the variance deconvolution estimator by considering a generic scalar QoI $Q$ which is a function of a vector of uncertain input parameters $\xi \in \Xi \subseteq \mathbb{R}^d$, where $d \in \mathbb{N}$ is the number of uncertain parameters that $Q$ depends on. As is typical for UQ, we consider the uncertain parameters to be described by a joint probability density function (PDF) $p(\xi)$ such that $\int_\Xi p(\xi) d\xi = 1$.

For UQ, we are interested in computing (central) moments of Q, such as
\begin{equation}\label{eq:UQmoments}
    \begin{split}
        \EE{Q}  &= \int_\Xi{ Q(\xi) \, p(\xi) \, \mathrm{d}\xi  } \quad \mathrm{and} \\
        \Var{Q} &= \int_\Xi{ \left( Q(\xi) - \EE{Q} \right)^2 p(\xi) \, \mathrm{d}\xi  },
    \end{split}
\end{equation}
with $\EE{Q}$ and $\Var{Q}$ denoting the mean and the variance of $Q$, respectively. These central moments can be approximated with Monte Carlo (MC) by drawing $\Nxi$ samples for the QoI, each of them corresponding to an independent sample of $\xi$ from its PDF and subsequent run of a possibly expensive computational code, as
\begin{equation} \label{eq:UQMC_moments}
\begin{split}
 \EE{Q}  &\approx \frac{1}{\Nxi} \sumxi Q( \xii )  \\
 \Var{Q} &\approx \frac{1}{\Nxi-1} \sumxi \left(Q( \xii ) - \frac{1}{\Nxi} \sum_{k=1}^{\Nxi} Q(\xi^{(k)})\right)^2.
 \end{split}
\end{equation}
Both estimators in Eq.~\eqref{eq:UQMC_moments} are unbiased, meaning that if we take the expectation of the MC estimators, they are exactly equal to the integral moments in Eq.~\eqref{eq:UQmoments}\footnote{For more details, the interested reader can refer to \cite{Owen}.}. 

In the case of a non-deterministic solver, our QoI is obtained as statistics of elementary and observable events. We consider observing a single elementary realization $f(\xi, \eta)$. We have introduced the random variable $\eta$, possibly a random vector, to notionally represent the code's stochastic behavior; unlike with $\xi$, knowledge of $\eta$ is neither implied nor required. 
In the case of a Monte Carlo radiation transport (MCRT)solver, $\eta$ describes the inaccessible vector of random variables used by the solver to generate the particle's random walk, and $f(\xi, \eta)$ is the result of a single particle history. One realization of QoI $Q(\xi)$ is obtained as an expected value of elementary events $f$ over multiple realizations of $\eta$, \begin{equation}
\label{eq:QoI_UQ}
 Q(\xi) = \EE{f(\xi,\eta) \;\middle|\; \xi}  \defin \EEeta{ f(\xi,\eta) };
\end{equation}
we have defined the shorthand notation $\mathbb{E}_X\left[\cdot\right]$ to indicate the expected value over realizations drawn with respect to the variable $X$, \textit{i.e.} with all non-$X$ variables fixed. In practice, the elementary event will have a finite number of realizations, so $Q(\xi)$ is approximated using a finite number of particles $\Neta$,
\begin{equation}\label{eq:qpoll}
    Q(\xii) \approx \frac{1}{\Neta} \sumeta \fij
          \defin \Qpoll\left(\xii\right).
\end{equation}
Combining Eqs.~\eqref{eq:UQMC_moments} and \eqref{eq:qpoll} provides approximations for the expected value of the QoI,
\begin{equation}\label{eq:MCMC_moments}
        \EE{Q} \approx \frac{1}{\Nxi} \sumxi \Qpoll(\xii)
        = \frac{1}{\Nxi} \sumxi \left( \frac{1}{\Neta} \sumeta \fij \right)  \defin \Qhat ,
\end{equation}
nesting the MC approximation for $Q(\xi)$ inside the MC approximations for the UQ statistics of interest. We can understand how the use of a stochastic solver impacts forward uncertainty propagation by evaluating features of this nested MC-MC UQ estimator. 


\subsection{Statistical properties of the MC-MC estimator}
The following evaluates statistical properties of the MC-MC estimator under the assumption that the number of histories $\Neta$ is constant for each UQ sample\footnote{This assumption is not strictly required, but simplifies derivations.}, such that the total estimator cost is $\mathcal{C} = \Nxi \times \Neta$. We first consider the effects of the nested sampling estimator by studying its bias, taking the expected value of the estimator over both spaces,
\begin{equation}\label{eq:EE-qhat}
    \begin{split}
        \EE{\Qhat} &= \EE{\frac{1}{\Nxi} \sumxi \left(\frac{1}{\Neta}\sumeta \fij \right)} \\
        &= \frac{1}{\Nxi} \sumxi \left(\frac{1}{\Neta} \sumeta \EExi{\EEeta{\fij}} \right) \\
        &= \frac{1}{\Nxi} \sumxi \left( \frac{1}{\Neta} \sumeta \EExi{Q(\xii)} \right) \\
        &= \frac{1}{\Nxi} \sumxi \EExi{Q(\xii)} = \EExi{Q(\xi)} .
    \end{split}
\end{equation}
By taking advantage of the linearity of the expected value operator, \textit{i.e.} $\EE{X_1 + X_2 + \cdots + X_j} = \EE{X_1} + \EE{X_2} + \cdots + \EE{X_j}$ \textcolor{red}{[cite]}, we see that the nested MC-MC sampling estimator for $\EE{Q}$ is unbiased. We next consider the variance of nested sampling estimator, using the known property of the variance of a linear combination \textcolor{red}{[cite]},
\begin{equation}\label{eq:Var-qhat}
    \begin{split}
        \Var{\Qhat} &= \Var{\frac{1}{\Nxi} \sumxi \left( \frac{1}{\Neta} \sumeta \fij \right)} \\
        &= \Var{\frac{1}{\Nxi} \sumxi \Qpoll(\xii)} \\
        &= \frac{1}{\Nxi^2} \sumxi \Var{\Qpoll(\xii)} \\
        &= \frac{1}{\Nxi} \Var{\Qpoll(\xi)} .
    \end{split}
\end{equation}
Because $\Qpoll(\xi)$ is the average of elementary observable events which depend on $\eta$, we can apply the law of total variance \textcolor{red}{[citation]} to $\Qpoll(\xi)$ to evaluate further:
\begin{equation}\label{eq:Var-qpoll}
    \begin{split}
        \Var{Y} &= \EE{\Var{Y | X}} + \Var{\EE{Y | X}} \\
        \rightarrow \Var{\Qpoll(\xi)} &= \EExi{\Veta{\Qpoll(\xi)}} + \Vxi{\EEeta{\Qpoll(\xi)}} \\
        &= \EExi{\Veta{\frac{1}{\Neta}\sumeta f(\etaj,\xi)}} + \Vxi{Q(\xi)} \\
        &= \EExi{\frac{1}{\Neta^2} \sumeta \Veta{f(\etaj,\xi)}} + \Vxi{Q(\xi)} \\
        &= \EExi{\frac{1}{\Neta} \Veta{\f}} + \Vxi{Q(\xi)} .
    \end{split}
\end{equation}
Combining Eqs.~\eqref{eq:Var-qhat} and~\eqref{eq:Var-qpoll}, the variance of the MC-MC estimator is
\begin{equation}\label{eq:Var-qhat-final}
    % \Var{\Qhat} = \frac{\EExi{\frac{1}{\Neta} \Veta{f}} + \Vxi{Q(\xi)}}{\Nxi} .
    \Var{\Qhat} = \frac{\EExi{\Veta{f}} + \Neta \Vxi{Q}}{\Nxi}
\end{equation}
These evaluations allow us to better understand the effects of the nested MC-MC estimator on UQ statistics. It is known that given population mean $\mu$, population variance $\sigma^2$, and sample mean $\Bar{X}$ over sample size $n$, $\EE{\Bar{X}} = \mu$ and $\Var{\Bar{X}} = \sigma^2/n$. If we think of the nested MC-MC estimator for $\EE{Q}$ as a nested sample mean with nested sample sizes $\Nxi$ and $\Neta$, it follows that $\EE{\Qhat}=\EE{Q}$. Similarly, it follows that $\Var{\Qhat} = \Var{\Qpoll}/\Nxi$. Perhaps the less intuitive result is that the variance of $\Qpoll$, also a sample estimator, decomposes into two distinct contributions: $\Vxi{Q(\xi)}$, which we may think of as the parametric (traditional UQ) variance of the QoI; and $\EExi{\frac{1}{\Neta} \Veta{\f}}$, which we may think of as the variance contribution from the stochastic solver.


\subsection{Variance deconvolution and practical implementation}
The typical sampling-based UQ workflow is to collect evaluations of the QoI over the UQ space $Q(\xi)$, then approximate $\EE{Q}$ and $\Var{Q}$ with MC estimators (Eq.~\eqref{eq:UQMC_moments}). When a stochastic solver is introduced, $Q(\xi)$ becomes inaccessible and may only be approximated by $\Qpoll(\xi)$. This does not present a problem in computing $\EE{Q}$, because as we've shown, $\EE{\Qhat}$ provides an unbiased estimate for $\EE{Q}$. For $\Var{Q}$, this is not the case -- Eq.~\eqref{eq:Var-qpoll} shows that the stochastic solver introduces a bias, $\EExi{\Veta{\f}}/\Neta$. Existing methods \textcolor{red}{[citation]} have used a brute-force approach to this issue, increasing the number of elementary event realizations $\Neta$ such that the noise contribution from the stochastic solver can be assumed to be negligible, approximating $\lim_{\Neta \to \infty} \Veta{\f} = 0$. In many application spaces, increasing $\Neta$ enough that this assumption holds can increase computational cost to the point of intractability. In that sense, perhaps the most important result of Eq.~\eqref{eq:Var-qpoll} is a practical path to a fully unbiased estimate of $\Var{Q}$. Given that $\Qpoll$ is an accessible quantity, $\Var{\Qpoll}$ is calculable by taking the variance of several evaluations of $\Qpoll(\xi)$. Similarly, $\Veta{\f}$ is calculable given $\Neta$ realizations of $\f$ per UQ sample, making $\EExi{\Veta{\f}}$ calculable by averaging $\Veta{\f}$ over the number of UQ samples. Re-arranging Eq.~\eqref{eq:Var-qpoll}, an unbiased estimate of $\Var{Q}$ can be estimated by removing the stochastic solver noise from the total polluted variance,
\begin{equation}\label{eq:var-deconv}
    \Var{Q} = \Vxi{Q} = \Var{\Qpoll} - \frac{\EExi{\SigSqeta}}{\Neta} ,
\end{equation}
where $\SigSqeta$ is defined as the variance of the histories for one fixed UQ parameter, \textit{i.e.} $\SigSqeta \defin \Veta{\f}$. To practically implement this method for computing $\Var{Q}$, we will use several sample-estimator counterparts,
\begin{equation}
    \begin{split}
        \Var{\Qpoll} &\approx \frac{1}{\Nxi-1}\sumxi\left( \Qpoll(\xii) - \Qhat \right)^2 \defin \tilde{S}^2 \\
        \SigSqeta(\xii) &\approx \frac{1}{\Nxi-1}\sumeta\left( \fij - \Qpoll(\xii) \right)^2 \defin \hatSigSqeta(\xii) \\
        \frac{\EExi{\SigSqeta}}{\Neta} &\approx \frac{1}{\Neta} \frac{1}{\Nxi} \sumxi \hatSigSqeta(\xii) \defin \muRT .
    \end{split}
\end{equation}
The sample estimator counterpart of Eq.~\eqref{eq:var-deconv} is then
\begin{equation} \label{eq:sampling-var-deconv}
 \Var{Q} \approx S^2 = \tilde{S}^2 - \muRT.
\end{equation}



\subsection{Prescribing a computational budget} % Start talking about optimization
The goal of a precise estimator is to obtain statistics with the lowest possible variance for a prescribed computational budget. In the case of this MC-MC estimator, assuming a linear cost model with a constant $\Neta$ for all $\Nxi$ UQ runs, the total computational budget is $\mathcal{C}=\Nxi \times \Neta$. Eq.~\eqref{eq:Var-qhat} suggests that the variance of the estimator, $\Var{\Qhat}$, is minimized when $\Var{\Qpoll}$ is; the most effective computational budget for $\Qhat$ is that which minimizes $\Var{\Qpoll}$. In practical implementation, the most effective computational budget for $\Qhat$ is that which minimizes the variance of $\Qpoll$'s sample estimator, $\tilde{S}^2$. \textcolor{red}{Note: Gianluca has suggested that I may want a more robust reasoning for minimizing $\Var{\tilde{S}^2}$ rather than $\Var{S^2}$.}
% Eq.~\eqref{eq:Var-qhat} also suggests that $\Var{\Qhat}$ is minimized when $\Neta=1$ (although computation of $\hatSigSqeta$ requires $\Neta \geq 2$).

\noindent Before presenting the derivation for minimizing $\Var{\tilde{S}^2}$, we introduce some notation:
\begin{equation}
    \begin{split}
        \mu\left[ X \right] &\defin \EE{ X } \\
        \mu_k\left[ X \right] &\defin \EE{ (X-\mu)^k } \\
        \mu_{\eta,k}\left[ X \right] &\defin \EEeta{ (X-\mu_\eta)^k } \\
        \sigma^2\left[ X \right] &= \mu_2\left[ X \right] .
    \end{split}
\end{equation}
Given that $\tilde{S}^2$ is a sample variance, its variance is \textcolor{red}{[cite]}:
\begin{equation}
    \Var{\tilde{S}^2} = \frac{\mu_4\left[\Qpoll\right]}{\Nxi} - \frac{ \sigma^4\left[\Qpoll\right](\Nxi-3) }{ \Nxi (\Nxi-1) }.
\end{equation}
Introducing the shorthand notation $\tilde{\mu}_k \defin \mu_k\left[\Qpoll\right]$, 
\begin{equation}\label{eq:var-s2-tilde}
    \Var{\tilde{S}^2} = \frac{\tilde{\mu}_4}{\Nxi} - \frac{ \tilde{\sigma}^4 (\Nxi-3) }{ \Nxi (\Nxi-1) }.
\end{equation}

Our objective is to study Eq.~\eqref{eq:var-s2-tilde} as a function of $\Neta$ to answer the question: Under a given cost constraint, how much should we resolve each stochastic code run to minimize $\Var{\tilde{S}^2}$, and therefore $\Var{\Qhat}$? We start by considering a simple linear cost model in which the start-up time is negligible, \textit{i.e.} $\mathcal{C}=\Nxi \times \Neta$ and we re-write Eq.~\eqref{eq:var-s2-tilde} as a function of $\Neta$ (while $\mathcal{C}$ is kept constant)
\begin{equation}
 \Var{\tilde{S}^2} = \Neta \frac{ \tilde{\mu}_4 }{ \mathcal{C} } - \frac{ \tilde{\sigma}^4 \Neta \left( \mathcal{C} - 3 \Neta \right) }{ \mathcal{C} \left( \mathcal{C} - \Neta \right) }.
\end{equation}
We want to solve $\partial \Var{\tilde{S}^2} / \partial \Neta = 0$, which is
\begin{equation}\label{eq:dVar-dNeta}
 \pdiff{ \Var{\tilde{S}^2} }{ \Neta } = \frac{ \tilde{\mu}_4 }{ \mathcal{C} } + \frac{ \Neta }{ \mathcal{C} } \pdiff{ \tilde{\mu}_4 }{ \Neta }
                                      - \pdiff{ \tilde{\sigma}^4 }{ \Neta } \frac{ \Neta \left( \mathcal{C} - 3 \Neta \right) }{ \mathcal{C} \left( \mathcal{C} - \Neta \right) }
                                      - \tilde{\sigma}^4 \pdiff{ \Neta \left( \mathcal{C} - 3 \Neta \right) }{ \mathcal{C} \left( \mathcal{C} - \Neta \right) },
\end{equation}
where the statistics of interest are  
\begin{equation} \label{eq:E_Q_tilde}
 \begin{split}
  \tilde{\mu}_{4} &= \EE{ \Qpoll^4 } - 4 \EE{ \Qpoll^3 } \EE{ Q } + 6 \EE{ \Qpoll^2 } \EE{Q}^2 - 3 \EE{ Q }^4   \\
  \EE{\tilde{Q}^4} &= \EE{ Q^4 } + \frac{1}{\Neta^4} \left[ 6 \Neta^3 \EExi{ Q^2 \SigSqeta } + 4 \Neta^2 \EExi{ Q \mu_{\eta,3}[f] } 
               + \Neta \EExi{ \mu_{\eta,4}[f] + 3 \left( \Neta-1 \right) (\SigSqeta)^2} \right] \\
  \EE{\tilde{Q}^3} &= \EE{Q^3} + \frac{3}{\Neta} \EExi{ Q \SigSqeta } + \frac{1}{\Neta^2} \EExi{ \mu_{\eta,3} } \\
  \EE{\tilde{Q}^2} &= \Var{\Qpoll} + \EExi{Q}^2 = \Vxi{Q} + \frac{ \EExi{ \SigSqeta } }{ \Neta } + \EExi{Q}^2 = \EExi{Q^2} + \frac{1}{\Neta}\EExi{\SigSqeta}\\
   \tilde{\sigma}^4 &= \Vxi{Q}^2 + 2 \Vxi{Q} \EExi{ \SigSqRT } + \EExi{ \SigSqRT }^2 = \left( \Var{ \Qpoll } \right)^2\\  
 \end{split}
\end{equation}
and the derivatives are
\begin{equation}\label{eq:dEQ_tilde-dNeta}
 \begin{split}
  \pdiff{ \tilde{\mu}_4 }{ \Neta } &= \pdiff{ \EE{\Qpoll^4} }{ \Neta } - 4 \EE{Q} \pdiff{ \EE{\Qpoll^3} }{ \Neta } + 6 \EE{Q}^2 \pdiff{ \EE{\Qpoll^2} }{ \Neta } \\
  \pdiff{ \EE{\tilde{Q}_4} }{ \Neta } &= - \frac{6}{\Neta^2} \EExi{ Q^2 \SigSqeta } - \frac{ 8 }{ \Neta^3 } \EExi{ Q \mu_{\eta,3} } - \frac{ 3 }{ \Neta^4 } \EExi{ \mu_{\eta,4} } 
                                      - \frac{3}{\Neta^3} \left( 2 - \frac{3}{\Neta} \right) \EExi{ \left( \SigSqeta \right)^2 } \\
  \pdiff{ \EE{\Qpoll^3} }{ \Neta } &= - \frac{ 3 }{ \Neta^2 } \EExi{ Q \SigSqeta } - \frac{ 2 }{ \Neta^3 } \EExi{ \mu_{\eta,3} } \\  
  \pdiff{ \EE{\Qpoll^2} }{ \Neta } &= - \frac{1}{\Neta^2} \EExi{ \SigSqeta } \\
  \pdiff{ \tilde{\sigma}^4 }{ \Neta }      &= - \frac{2}{\Neta^2} \Vxi{Q} \EExi{\SigSqeta} - \frac{2}{\Neta^3} \left( \EExi{ \SigSqeta } \right)^2 = - \frac{2}{\Neta^2} \left( \Vxi{Q} + \frac{ \EExi{\SigSqeta} }{ \Neta } \right)\\
  \pdiff{ \left(\Neta \left( \mathcal{C} - 3 \Neta \right) \right) }{ \left( \mathcal{C} \left( \mathcal{C} - \Neta \right) \right) } &= \frac{ \mathcal{C}^2 - 6 \mathcal{C} \Neta + 3 \Neta^2}{ \mathcal{C} \left( \mathcal{C} - \Neta \right)^2 }.
 \end{split}
\end{equation}

\noindent The previous expressions suggest which statistics one needs to compute in order to solve the resource allocation problem.

\subsubsection{Unbiased estimators for analytical terms}
In practical application, one would aim to run a pilot study to compute the necessary terms needed to find the optimal cost configuration for a subsequent UQ study. After developing the analytic expressions (Eq.~\eqref{eq:dVar-dNeta},~\eqref{eq:E_Q_tilde}, and~\eqref{eq:dEQ_tilde-dNeta}) we need to find unbiased estimators to compute these analytic terms from available tallies.

We introduce some notation for a (biased) sample central moment:
\begin{equation}
    m_k[X] = \frac{1}{N}\sum_{i=1}^N \left(x_i - m\right)^k
\end{equation}
where $m$ is the (unbiased) sample central mean. We use the notation $\hat{\cdot}$ to indicate a sample estimator. The unbiased central moments over $\Neta$ are\footnote{See appendix for full derivations of unbiased central moments.}:
\begin{equation}
    \begin{split}
        \hat{\sigma}_{\eta}^2 &= \frac{\Neta}{\Neta-1} m_{\eta,2} \\
        \hat{\mu}_{\eta,3} &= \frac{\Neta^2}{(\Neta-1)(\Neta-2)} \\
        \hat{\mu}_{\eta,4} &= \frac{\Neta\left[ \left(\Neta^2 - 2\Neta + 3\right)m_{\eta,4} - 3\left(2\Neta-3\right)m_{\eta,2}^2 \right]}{(\Neta-1)(\Neta-2)(\Neta-3)} \\
        \hat{\sigma}_\eta^4 &= \frac{\Neta\left[ \left(\Neta^2 - 3\Neta + 3\right)m_{\eta,2}^2 - \left(\Neta-1\right)m_{\eta,4} \right]}{(\Neta-1)(\Neta-2)(\Neta-3)}
    \end{split}
\end{equation} such that 
\begin{equation}
    \begin{split}
        \EE{\hat{\sigma}_{\eta}^2} &= \EExi{\SigSqeta}, \\
        \EE{\hat{\mu}_{\eta,3}} &= \EExi{\mu_{\eta,3}}, \\
        \EE{\hat{\mu}_{\eta,4}} &= \EExi{\mu_{\eta,4}}, \text{and}\\
        \EE{\hat{\sigma}_\eta^4} &= \EExi{\sigma_\eta^4} .
    \end{split}
\end{equation}
Equations \eqref{eq:E_Q_tilde} give us unbiased estimators for $\EE{Q^2}$, $\EE{Q^3}$, and $\EE{Q^4}$, leaving us needing to compute estimators for $\EE{Q\SigSqeta}$, $\EE{Q\mu_{\eta,3}}$, and $\EE{Q^2\SigSqeta}$ from the available $\Qpoll \hatSigSqeta$, $\Qpoll \hat{\mu}_{\eta,3}$, and $\Qpoll^2 \hatSigSqeta$. \textcolor{red}{Note - what level of detail should I include here? I have the full derivations, but I don't know how much to include / how much people will want to see.}
\begin{equation}
    \begin{split}
        m_{\eta,2} &= \frac{1}{\Neta}\sumeta \left(f-\Qpoll\right)^2 = \frac{1}{\Neta}\left(\sumeta f^2 - \Neta\Qpoll^2\right) \\
        \hatSigSqeta &= \frac{1}{\Neta-1}\left(\sumeta f^2 - \Neta\Qpoll^2\right) \\
        \EEeta{\Qpoll\hatSigSqeta} &= \frac{1}{\Neta-1}\left( \EEeta{\Qpoll\sumeta f^2} - \Neta\EEeta{\Qpoll^3}\right) \\
        &= ... = Q\SigSqeta + \frac{1}{\Neta}\mu_{\eta,3} \\
        \EExi{Q\SigSqeta} &= \EE{\Qpoll \hatSigSqeta} - \frac{1}{\Neta}\EE{\hat{\mu}_{\eta,3}}
    \end{split}
\end{equation}
\begin{equation}
    \begin{split}
        m_{\eta,3} &= \frac{1}{\Neta}\sum_{j=1}^\Neta \left(f-\Qpoll\right)^3 = \frac{1}{\Neta}\left(\sumeta f^3 - 3\Qpoll\sumeta f^2 + 2\Neta\Qpoll^3 \right) \\
        \hat{\mu}_{\eta,3} &= \frac{\Neta}{(\Neta-1)(\Neta-2)}\left(\sumeta f^3 - 3\Qpoll\sumeta f^2 + 2\Neta\Qpoll^3 \right) \\
        \EEeta{\Qpoll\hat{\mu}_{\eta,3}} &= \frac{\Neta}{(\Neta-1)(\Neta-2)} \left( \EEeta{\Qpoll\sumeta f^3} - 3\EEeta{\Qpoll^2\sumeta f^2} + 2\Neta\EEeta{\Qpoll^4}\right) \\
        &= ... = Q\mu_{\eta,3} + \frac{1}{\Neta}\mu_{\eta,4} + \frac{2-3\Neta}{\Neta(\Neta-2)}(\SigSqeta)^2\\
        \EExi{Q\mu_{\eta,3}} &= \EE{\Qpoll \hat{\mu}_{\eta,3}} - \frac{1}{\Neta}\EE{\hat{\mu}_{\eta,4}} - \frac{2-3\Neta}{\Neta(\Neta-2)}\EExi{(\hatSigSqeta)^2}
    \end{split}
\end{equation}


\begin{equation}
    \begin{split}
        m_{\eta,2} &= \frac{1}{\Neta}\sum_{j=1}^\Neta \left(f-\Qpoll\right)^2 = \frac{1}{\Neta}\left(\sumeta f^2 - \Neta\Qpoll^2\right) \\
        \hatSigSqeta &= \frac{1}{\Neta-1}\left(\sumeta f^2 - \Neta\Qpoll^2\right) \\
        \EEeta{\Qpoll^2\hatSigSqeta} &= \frac{1}{\Neta-1}\left( \EEeta{\Qpoll^2\sumeta f^2} - \Neta\EEeta{\Qpoll^4}\right) \\
        &= ... = Q^2\SigSqeta + \frac{2}{\Neta}Q\mu_{\eta,3} + \frac{1}{\Neta^2}\mu_{\eta,4} + \frac{\Neta-3}{\Neta^2}(\SigSqeta)^2\\
        \EExi{Q^2\SigSqeta} &= \EE{\Qpoll^2 \hatSigSqeta} - \frac{2}{\Neta}\EE{\Qpoll \hat{\mu}_{\eta,3}} + \frac{1}{\Neta^2}\EE{\hat{\mu}_{\eta,4}} - \frac{(\Neta+1)^2}{\Neta^2(\Neta-2)}\EE{(\hatSigSqeta)^2}
    \end{split}
\end{equation}


To conduct a pilot study, the workflow is to:
\begin{enumerate}
    \item For a single $\xii$, run a MC simulation with $\Neta$ histories.
    \item Compute $\hat{\mu}_{\eta,3,i}$, $\hat{\mu}_{\eta,4,i}$, and $\hat{\sigma}_{\eta,i}^4$ using the equations above, in addition to $\hat{\sigma}_{\eta,i}^2$ and $\Qpoll_i$.
    \item After all $\Nxi$ have run, average $\hat{\mu}_{\eta,3,i}$, $\hat{\mu}_{\eta,4,i}$, $\hat{\sigma}_{\eta,i}^4$, $\hat{\sigma}_{\eta,i}^2$, and $\Qpoll_i$ over $\Nxi$ for unbiased estimates of $\EExi{\SigSqeta}$, $\EExi{\mu_{\eta,3}}$, $\EExi{\mu_{\eta,4}}$, $\EExi{(\SigSqeta)^2}$, and $\EE{Q}$.
    \item Use the equations above to compute unbiased estimates of $\EE{Q^2}$, $\EE{Q^3}$, $\EE{Q^4}$, $\EE{Q\SigSqeta}$, $\EE{Q\mu_{\eta,3}}$, and $\EE{Q^2\SigSqeta}$.
    \item Pass these terms to a function that will compute $\tilde{\mu}_4$, $\pdiff{ \tilde{\mu}_4 }{ \Neta }$, $\tilde{\sigma}^4$, and $\pdiff{\tilde{\sigma}^4}{\Neta}$ to optimize $\pdiff{ \Var{\tilde{S}^2} }{ \Neta }$ as a function of $\Neta$.
\end{enumerate}
From here, one would ideally have computed the optimum $\Neta$ at which to run a full UQ study, and assign the number of UQ samples in accordance with the prescribed total estimator cost. 



\section{Global sensitivity analysis for stochastic solvers}
\subsection{Straightforward Saltelli}
\todo{From submitted M\&C extended abstract.}
Let's consider a generic QoI $Q$, which expresses a mapping from the vector of $d$ input parameters $\xi \in \Xi \subset \mathbb{R}^d$, with joint probability density function (PDF) $p(\xi)$, as $Q = Q(\xi)$. In standard UQ workflows, we are concerned with estimating statistics for $Q$ with respect to the input parameters, \textit{e.g.} moments like the mean and variance:
\begin{equation}
\label{eq:UQmoments}
%  \begin{split}
 \EExi{Q} \defin \int_\Xi Q(\xi) p(\xi) d\xi \quad \mathrm{and} \quad 
 \Vxi{Q} \defin \int_\Xi \left( Q(\xi) - \EExi{Q} \right)^2 p(\xi) d\xi.
%  \end{split}
\end{equation}

In GSA\footnote{We limit ourselves to variance-based decomposition strategies here, although other approaches are also possible.}, %~\cite{OwenHigh,GeraciCMAME}}, 
we are interested in determining how each parameter $\xi_i$, or their combinations, affects the variance. The ANalysis Of VAriance (ANOVA) approach, introduced by Sobol' in the seminal paper~\cite{sobol}, is the most adopted method. % and additional details will be provided in the full paper. 
Sobol's decomposition allows us to write the variance of $Q$ as the sum of conditional contributions~\cite{Crestaux2009}
\begin{equation}\label{eq:conditional-var}
 \Vxi{Q} = \sum_{\substack{u \subseteq \left\{ 1, 2, \dots, d \right\} \\ u \neq \emptyset}} V_u, \quad \mathrm{where} \quad 
% \end{equation}
% where
% \begin{equation}
 V_u = \Var{ \EE{ Q | \xi_u } } - \sum_{ \substack{ v \subset u, v \neq u,  v \neq \emptyset } } V_v,
\end{equation}
where we use $u$ to indicate a set of indices that define $\xi_u = \left[ \xi_{u_1}, \dots, \xi_{u_s} \right]^{\mathrm{T}}$, where $s = \mathrm{card}(u)$. It follows that the Sobol' indices can be defined as 
\begin{equation}\label{eq:si}
 S_u = \frac{V_u}{\Vxi{Q}}, \quad \mathrm{with} \quad 
 \sum_{ \substack{ u \subseteq \left\{ 1, 2, \dots, d \right\} \\ u \neq \emptyset } } S_u = 1,
\end{equation}
where, if $s=1$, we refer to $S_u$ as first-order sensitivity index (SI).
% We are usually interested in the first-order sensitivity indices ($s=1$), which measure the effect of each single variable or the total indices, which aggregate the effect of all sensitivity indices in which a specific variable is contained.
%
% In the context of MC RT, we embed the concept of variance deconvolution in all the GSA tasks. The interested reader should refer to our previous work, \textit{e.g.}~\cite{ClementsCSRI2021} for more details, although an highlight of this is introduced in the following. 
In MC RT applications, %the presence of non-deterministic solvers, 
the QoI $Q$ can only be approximated by averaging a number of realizations (see~\cite{ClementsANS2022} for details). For instance, if we indicate the elementary particle history event with $f$, $Q(\xi)$ can be thought of as
\begin{equation}
 Q(\xi) \defin \EEeta{ f(\xi,\eta) } \approx \frac{1}{\Neta} \sum_{j=1}^{\Neta} f( \xi, \eta^{(j)} ) \defin \Qpoll(\xi).
\end{equation}
% where we explicitly highlighted how the finite number of particle histories $\Neta$ introduces an approximation.
The additional variable $\eta$ is introduced only to notionally represent the randomness of a MC RT solver. In practice, $\eta$, unlike $\xi$, is not controlled and merely reflects that multiple particle histories, even if for the same system defined by $\xi$, will follow different trajectories. The variance deconvolution consists of %applying the law-of-total variance for 
obtaining an approximation of the parametric variance $\Vxi{Q}$ from observable quantities via%This relationship can be expressed as
\begin{equation}\label{eq:var-deconv}
 \Vxi{Q} = \Var{\Qpoll} - \frac{\EExi{ \SigSqeta }}{\Neta},
\end{equation}
where $\Var{\Qpoll}$ represents the total variance (corrupted by the MC RT noise) and $\EExi{ \SigSqeta }$ represents the average contribution of the solver's stochasticity $
\SigSqeta\defin \Veta{ f }$. %This tool can be used for both sampling- and PCE-based GSA approaches. 
% 
% construction (which in turn can be used for GSA). These details are briefly hinted in the following, while
%Additional details will be provided in the full manuscript.

\subsection{Existing methods for GSA with stochastic solvers}
\subsection{Variance deconvolution appended to Saltelli}
\todo{From submitted M\&C extended abstract.}
The widely-adopted ANOVA method %introduced by Saltelli~\cite{Saltelli} 
assumes a deterministic solver, therefore not taking into account the additional stochasticity introduced by MC solvers. In~\cite{OlsonANSWinter}, we incorporated variance deconvolution to %into the Saltelli method, 
provide an unbiased sampling-based method for computing SI when using an MC RT solver. %This is summarized below.
%
% From the definition of Sobol' indices in Eq.~\ref{eq:si}, we see that we need $\Vxi{Q}$ and  $V_u \forall u \in d$, where $V_u$ is the conditional variance $\Var{ \EE{ Q | \xi_u } }$. To further understand the conditional variance of UQ parameter $\xi_u$ from our variance deconvolution approach, we can break $\xi$ down into its components $\xi_u, u=1,2,..,d$ such that $Q\left(\xi\right) = Q\left( \xi_u, \xi_{\sim u} \right)$. 
If we introduce additional notation for the expectation of the QoI as a function of just a variable of interest $\xi_u$, we can write
\begin{equation}
 P(\xi_u) \defin \EEnu{ Q(\xiu,\xinu) } \approx \frac{1}{\Nsi} \sum_{k=1}^{\Nsi} \Qpoll( \xiu, \xinu^{(k)} ) \defin \Ppoll(\xiu),
\end{equation}
and we can express Eqs.~\eqref{eq:conditional-var} and \eqref{eq:si} as 
% Eq.~\ref{eq:var-deconv} can be extended to deconvolve $\Var{\Ppoll}$ into contributions from $\xiu$ and $\xinu$:
\begin{equation}
 V_u %= \Var{\EE{Q | \xiu}} = \Vu{P} 
     = \Var{\Ppoll} - \frac{\EEu{ \Vnu{\Qpoll} }}{\Nsi} \quad \mathrm{and} \quad 
% \end{equation}
% 
% Applying this notation, the main effect SI of $\xi_u$ can be written as 
% \begin{equation}\label{eq:si-deconv}
 S_u %= \frac{V_u}{\Vxi{Q}} 
     = \frac{\Var{\Ppoll} - \frac{\EEu{ \Vnu{\Qpoll} }}{\Nsi}}{\Var{\Qpoll} - \frac{\EExi{ \SigSqeta }}{\Neta}} .
\end{equation}

% To calculate the denominator terms terms in Eq.~\ref{eq:si-deconv}, we split the total desired estimator cost $C$ into a number of UQ samples $\Nxi$ and a number of histories per UQ sample $\Neta$ such that $C = \Nxi \times \Neta$. For each UQ sample $\xi_i, i=1,..,\Nxi$, we calculate the polluted QoI $\Qpoll \left( \xi_i \right)$ and solver variance $\SigSqeta \left( \xi_i \right)$. After all UQ samples are complete, calculate the variance of the polluted samples $ \Var{\Qpoll}$ and average contribution of the solver's stochasticity $\EExi{ \SigSqeta }$. 
% The algorithm for calculating $V_u$ is similar, but requires further decomposition in that we must hold parameter $\xiu$ constant while re-sampling all other UQ parameters $\xinu$ $N_{SI}$ times, to compute $\Vnu{\Qpoll}$. This must be done for each of the main and total effect SIs, meaning this process must be completed $2d$ times to obtain all $S_u$ and $S_{Tu}$. Incorporating this into the total estimator cost, $C = \Nxi \times N_{SI} \times 2d \times \Neta$. 
Computing $V_u$ requires holding parameter $\xiu$ constant while re-sampling all other UQ parameters $\xinu$ $N_{SI}$ times. %to compute $\Vnu{\Qpoll}$. The need for evaluating the $2d$ $S_u$ and $S_{Tu}$ terms reflects in a total computational cost 
% This must be done for each of the main and total effect SIs, meaning this process must be completed $2d$ times to obtain all $S_u$ and $S_{Tu}$. Incorporating this into the total estimator cost, 
For $d$ main and $d$ total effects, the total estimator cost is $C = \Nxi \times N_{SI} \times 2d \times \Neta$. 




% Note - if third section ends up being GSA using surrogate models, it will follow a similar structure to the GSA section above ^
\section{Challenge problem application}
\subsection{Benchmark deterministic problem}

\section{Foreseen challenges}
