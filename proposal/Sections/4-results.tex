
\chapter{Preliminary results} \label{ch-results}

\section{Variance deconvolution for uncertainty quantification}
\subsection{Previous work}
\begin{itemize}
    \item Preliminary work by Aaron Olson used a version of this solver that used only 2 histories per UQ sample, assuming this was the most efficient/accurate way, and slightly inaccurately estimated the total and solver variance in a way that cancelled each other out. 
    \item As an intern at Sandia in collaboration with Aaron and Gianluca, I helped to develop the theory and corroborating numerical results for a more accurate, efficient, and broadly applicable estimator. This was published in the sandia csri proceedings (finished August 2021, published January 2022) with theory, numerical examples, and comparison with the previous method. 
    \item As further work as a Sandia intern and in collaboration with Aaron and Gianluca, I ran numerical studies to see if a trend existed in the variance of the estimator and found that one did. These results were significant because they suggest that the estimator can be optimized for use with various problems. This was published as a 4 page transactions in the proceedings for ANS Winter 2021, on which I am first author.
    \item Current work on this (ongoing since June-ish 2022), primarily in conjunction with Gianluca, has been the development of a closed-form solution for how to optimize the estimator. The goal is this:
    \begin{enumerate}
        \item Solve for closed-form solution of $\frac{\partial \mathbb{V}ar[\tilde{S}^2]}{\partial N_\eta}$
        \item Run a pilot study to numerically compute all terms in this closed-form solution
        \item Plug these numerically computed terms into an optimization function like python.scipy.brentq to find the minimum of this function, ie the location where the derivative = 0, to find where the variance of the estimator is minimized as a function of $N_\eta$
        \item Use the optimum configuration to actually run the problem, therefore receiving a UQ estimate with the lowest variance of all possible configurations
    \end{enumerate}
    The actual process has been:
    \begin{enumerate}
        \item I ran numerical experiments for a number of total estimator costs to see if the trend was universal, and it appeared to be. Ran atten-only and scattering test cases for a 1D slab. 
        \item Gianluca finished a closed-form expression for the derivative $\frac{\partial \mathbb{V}ar[\tilde{S}^2]}{\partial N_\eta}$ and attempted to use that closed-form solution to plot the same trendline that I found numerically. It was close, but slightly off. 
        \item I separately went through the derivations for the same closed-form expression for the derivative and found a coefficient that disagreed with one of Gianluca's; this fixed the issue, and we confirmed that the closed-form solution agreed with the existing numerical results. 
        \item Gianluca applied this closed-form expression to the test 1D slab with 3 material sections we'd been using to get an analytic solution for all of the necessary terms in the derivative, and wrote a Python function to find the minimum of this expression using the calculated terms. This optimizer computed a minimum that agreed with the existing numerical results. 
        \item Current status: The goal is to be able to run a pilot study and correctly calculate the optimum, meaning we need to be able to numerically compute these terms. I've written out derivations to check for bias in the numerically accessible terms compared to the terms in the closed-form derivative expression, and tried correcting some of the biases, but still am not able to use numerical results for the derivative terms to calculate an optimum that matches our existing results. 
    \end{enumerate}
\end{itemize}
\subsection{Goal: journal article}
I'm in the process of writing a journal article on this variance deconvolution method. It will include all theory, application algorithms, and numerical results up to this point, and application on a larger test problem. I'm hoping that I'll be able to get the numerical optimization calculation working such that it can be included in the journal article, but finishing the article is a higher priority than getting that perfectly working. If I can, fantastic, but if not we'll include the analytic results we have for it and the general idea for the numerical algorithm so we don't get `scooped', as Gianluca might say. 




\subsection{MCDC}
During the CEMeNT `Hackathon' in early October 2022, I began the process of adding this variance deconvolution to MCDC. I made some progress with adding in the frameworks, and don't anticipate it being computationally difficult, but likely time-intensive just in the nature of parsing and contributing to the existing MCDC framework that Ilham's built. 



\section{Global sensitivity analysis}
\subsection{Previous work}
Second author on paper submitted to ANS winter 2022, on using the variance deconvolution approach to accurately compute main and total sobol' indices for a problem with stochastic mixing and uncertain total cross sections.
\subsection{Current ongoing work}
Submitted extended abstract for M\&C paper that begins exploring this work and incorporates comparison to PCE approach for Si computation (as opposed to the sampling approach I've been discussing), in which a surrogate linear model is built and the surrogate coefficients are used to compute the sobol indices. However, the kicker is of course that the solver used for coefficient construction is also stochastic. The scope of this M\&C paper is to further study GSA using a stochastic solver, and I plan to expand it to a journal paper comprehensively looking at different GSA approaches applied to problems with stochastic solvers.

